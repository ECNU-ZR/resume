% !TEX program = xelatex

\documentclass{resume}
%\usepackage{zh_CN-Adobefonts_external} % Simplified Chinese Support using external fonts (./fonts/zh_CN-Adobe/)
%\usepackage{zh_CN-Adobefonts_internal} % Simplified Chinese Support using system fonts

\begin{document}
\pagenumbering{gobble} % suppress displaying page number

\name{Rui Zhou}

\basicInfo{
  \email{zhour2@student.unimelb.edu.au} \textperiodcentered\ 
  \phone{(+61) 0478663333} \textperiodcentered\ }

\section{\faGraduationCap\ Education}
\datedsubsection{\textbf{University of Melbourne}, Melbourne, Australia}{2018 -- Present}
\textit{Master student} in Information Technology (IT)\\
\textit{Weight. Average} 89/100
\datedsubsection{\textbf{East China Normal University (ECNU)}, Shanghai, China}{2013 -- 2017}
\textit{Bachelor student} in Computer Science and Technology (CS)\\
\textit{GPA} 3.54/4.0

% \section{\faUsers\ Experience}
% \datedsubsection{\textbf{NER Project}}{June. 2018 -- Present}
% \role{Java/Python}{Supervised by Prof. Rui Zhang}
% Brief introduction: Identify the names of people on academic pages
% \begin{itemize}
%   \item Used three dimensions to tag a name
%   \item Currently use Stanford NER model to train and test data
% \end{itemize}

% \section{\faUsers\ Experience}
% \datedsubsection{\textbf{Nim Game Project}}{Mar. 2018 -- May. 2018}
% \role{Java}{Individual Projects}
% Brief introduction: A Nim game system that allows human and AI to play
% \begin{itemize}
%   \item Contained two types of Nim game
%   \item Contained AI players
% \end{itemize}

\section{\faBriefcase\ Internship}

\datedsubsection{\textbf{ByteDance} Beijing, China}{Nov. 2018 -- Feb. 2019}
Brief introduction: Development of automated analysis diagnostic tool


\section{\faUsers\ Experience}

\datedsubsection{\textbf{Fault-tolerant Key/Value Service}}{Mar. 2019 -- May. 2019}
\role{Golang}{Individual Project}
\begin{itemize}
  \item Used Raft algorithm as the consensus algorithm
  \item Allow up to \textit{f} server failure within \textit{2f + 1} servers
  \item Provide linearizability for client-side
\end{itemize}


\datedsubsection{\textbf{Distributed Scrabble Game}}{Sep. 2018 -- Oct. 2018}
\role{Java}{Individual Project}
Brief introduction: A distributed sracbble game which allows multiple players to play on different PC


\datedsubsection{\textbf{Deceptive Reinforcement Learning}}{Mar. 2019 -- May. 2019}
\role{Python}{Supervised by Prof. Tim Miller}
Brief introduction: Use Q-Learning to solve the deceptive path-planning problem


\datedsubsection{\textbf{Automatic Fact Verification}}{Mar. 2019 -- May. 2019}
\role{Python}{Team Project}
Brief introduction: Fact Extraction and VERification (FEVER)
\begin{itemize}
  \item Ranked 3rd out of 160 teams
  \item Label Accuracy $69.7\%$, Sentence Selection F1 $71.1\%$
\end{itemize}


\datedsubsection{\textbf{Pacman Capture the Flag}}{Sep. 2018 -- Oct. 2018}
\role{Python}{Team Project}
Brief introduction: An Implementation of Pacman AI (Pacman Capture the Flag competition)
\begin{itemize}
  \item Ranked 19th out of 156 teams
\end{itemize}

% \datedsubsection{\textbf{Distributed Dictionary Project}}{Aug. 2018 -- Sep. 2018}
% \role{Java}{Individual Project}
% Brief introduction: A distributed dictionary system of client server model
% \begin{itemize}
%   \item Used multithreading on the server side to serve multiple clients
%   \item Used socket for the communication between client and server
% \end{itemize}

% \datedsubsection{\textbf{Cicso} Shanghai, China}{2016 -- 2016}
% \role{Intern}{}
% Brief introduction: Product testing

% \datedsubsection{\textbf{Gobang AI Project}}{Oct. 2015 -- Nov. 2015}
% \role{C++}{Individual Projects}
% Brief introduction: An Implementation of Gobang AI
% \begin{itemize}
%   \item Used MinMax Search and alpha–beta pruning
% \end{itemize}

% \datedsubsection{\textbf{DiskToy Project}}{Sep. 2015 -- Nov. 2015}
% \role{C++}{Collaborated with Chenyang Liu}
% Brief introduction: A tool used to anaylse disk usage
% \begin{itemize}
%   \item Used DFS to explore all dictionaries and files in given path
%   \item Used the graphical form of the Treemap to show the results
% \end{itemize}

% Reference Test
%\datedsubsection{\textbf{Paper Title\cite{zaharia2012resilient}}}{May. 2015}
%An xxx optimized for xxx\cite{verma2015large}
%\begin{itemize}
%  \item main contribution
%\end{itemize}

\section{\faHeartO\ Honors and Awards}
\datedline{Dean's Honours List 2018}{May. 2019}
\datedline{\textit{Bronze Medal}, Award on The ACM-ICPC Asia Regional Contest EC-final Site 2015}{Dec. 2015}
\datedline{\textit{Bronze Medal}, Award on The ACM-ICPC Asia Regional Contest Changchun Site 2015}{Oct. 2015}
\datedline{\textit{Bronze Medal}, Award on The ACM-ICPC Asia Regional Contest Shanghai Metropolitan Site 2015}{Oct. 2015}
\datedline{\textit{Bronze Medal}, Award on The ACM-ICPC Asia Regional Contest Shanghai Site 2014}{Dec. 2014}
\datedline{\textit{Bronze Medal}, Award on The ACM-ICPC Asia Regional Contest Shanghai Site Invitational 2014}{Jul. 2014}

\section{\faCogs\ Programming Skills}
\begin{itemize}[parsep=0.5ex]
  \item Programming Languages: C, C++, Java, Python
  \item Development: Distributed System, AI, Reinforcement Learning, Stream computing
\end{itemize}

\section{\faInfo\ Language Skills}
\begin{itemize}[parsep=0.5ex]
  \item ITLTS Listening: 7.0 Reading: 7.0 Writing: 5.5 Speaking: 5.5 Overall: 6.5
\end{itemize}

%\section{\faStar\ Preference Aspects}
%\begin{itemize}[parsep=0.5ex]
%  \item Data mining
%  \item Machine learning
%  \item Artificial intelligence
%\end{itemize}

%% Reference
%\newpage
%\bibliographystyle{IEEETran}
%\bibliography{mycite}
\end{document}
